\documentclass[a4paper,12pt]{article}

% -------------------------------------------------
% Pacchetti essenziali
% -------------------------------------------------
\usepackage[utf8]{inputenc}
\usepackage[T1]{fontenc}
\usepackage{lmodern}
\usepackage{amsmath,amsfonts,amssymb}
\usepackage{graphicx}
\usepackage{listings,xcolor}
\usepackage{enumitem}
\usepackage{hyperref}
\usepackage{tikz}
\usepackage[official]{eurosym}
\usepackage{float}
\hypersetup{
	colorlinks=true,
	linkcolor=blue,
	urlcolor=blue,
	citecolor=blue
}
% -------------------------------------------------

\begin{document}
	
	% ---------- Frontespizio (pag. 1) ----------------
	\title{\textbf{Stochastic Optimization}\\
		\vspace{0.5em}\huge\textbf{Report}
		\author{
			Di Battista Simona — 302689%\\
			\and
			Rostagno Andrea — 349152\\
		}
		
		\vspace{0.5cm}
		\large Assignment 2024/25}
	%%\date{\today}
	\maketitle
	\thispagestyle{empty}   % (facoltativo) niente numero in frontespizio
	\newpage                % <-- salto di pagina: l’indice parte da qui
	
	
	% ---------- Indice (pag. 2) ----------------------
	\pagenumbering{roman}   % numeri romani per indice (i, ii, iii…)
	\tableofcontents
	\newpage                % nuovo salto: inizia il testo
	
	% ---------- Testo principale (da pag. 3) ---------
	\pagenumbering{arabic}  % riparte da 1 con numeri arabi
	
	% =================================================
	\section{Introduction}
	Stochastic optimization addresses decision-making problems under uncertainty, where parameters not known in advance are modeled as random variables. Unlike deterministic optimization, stochastic optimization aims to identify optimal decisions by considering the probabilistic distribution of future scenarios. In this project, we specifically focus on two classical stochastic problems: the Newsvendor problem and the Assemble-to-Order (ATO) problem. The Newsvendor problem involves determining the optimal quantity of newspapers to order under uncertain demand, to maximize expected profits. In contrast, the ATO problem addresses component inventory and assembly decisions to satisfy uncertain demand for multiple products.\\
	
	First, to solve each of the two problems, scenarios were generated through a predefined scenario generation code provided beforehand. Subsequently, to reduce computational complexity, were applied two scenario reduction methods: 
	\begin{itemize}
		\item $K$-means clustering; 
		\item an heuristic method, based on Wasserstein distance.
	\end{itemize}
	The results obtained after the reduction were compared with those obtained before the reduction, to assess the effectiveness and stability of each strategy.
	
	\newpage
	\section{Problems explanation}
	
	In this section, are presented the two stochastic optimization problems analyzed in this report: the Newsvendor problem and the Assemble-to-Order (ATO) problem. Both are classical examples of two-stage stochastic programs, characterized by decisions that must be made before uncertain demand is revealed. First, each problem is introduced and mathematically formulated, then,  in the next sections, they are solved and analyzed using different scenario generation and reduction methods.
	
	\subsection{Newsvendor Problem}
	
	The Newsvendor problem is a classical example in stochastic optimization used to determine the optimal inventory level when demand is uncertain. In our analysis, a vendor must decide the optimal number of newspapers to buy at the beginning of a day, without knowing the exact daily demand, with the goal of maximizing total expected profit.
	
	%	\newpage
	\subsubsection{Problem Formulation}
	
	Formally, the problem is formulated as follows:
	\[
	\max_{x} \mathbb{E}[p\min(D(\omega), x) - cx]
	\]
	
	with:
	\begin{itemize}
		\item \( x \): decision variable representing the number of newspapers to buy;
		\item \( D(\omega) \): a random variable modeling daily demand, characterized by discrete scenarios \( d_s \), each with probability \( \pi_s \);
		\item \( p \): selling price per newspaper;
		\item \( c \): cost per newspaper.
	\end{itemize}
	\vspace{0.20cm}
	
	In our specific implementation:
	
	\begin{itemize}
		\item \( c = 1 \);
		\item \( p = 10 \),
	\end{itemize}
	
	\noindent with the parameters chosen by taking an example given in class.\\
	
	\noindent The model is implemented using the following integer linear programming formulation:
	\[
	\begin{aligned}
		\max & \quad p \sum_{s \in S} \pi_s y_s - c x \\
		\text{s.t.} & \quad y_s \leq x, & \forall s \in S \\
		& \quad y_s \leq d_s, & \forall s \in S \\
		& \quad x, y_s \geq 0 \quad \text{and integers}, & \forall s \in S
	\end{aligned}
	\]
	
	\noindent	where \( y_s \) represents the actual number of newspapers sold if scenario \( s \in S\) is realized (where $S$ is the set of possible scenarios).
	
	\subsection{Assemble-to-Order (ATO) Problem}
	
	The Assemble-to-Order (ATO) problem addresses decision-making in manufacturing systems, where final products are assembled from a set of pre-produced components once customer orders are realized. This two-stage stochastic program involves:
	\begin{itemize}
		\item \textbf{first stage}: decide the quantities of components to produce;
		\item \textbf{second stage}: determine the assembly quantities of final products, once demand is known.
	\end{itemize}
	
	\subsubsection{Problem Formulation}
	
	The mathematical formulation of the ATO problem is given by the following model:
	\[
	\begin{aligned}
		\max & \quad -\sum_{i \in \mathcal{I}} C_i x_i + \mathbb{E}\left[\sum_{j \in \mathcal{J}} P_j y_j(\omega)\right] \\
		\text{s.t.} & \quad \sum_{i \in \mathcal{I}} T_{im} x_i \leq L_m, & \forall m \in \mathcal{M} \\
		& \quad y_j(\omega) \leq d_j(\omega), & \forall j \in \mathcal{J}, \forall \omega \in \Omega \\
		& \quad \sum_{j \in \mathcal{J}} G_{ij} y_j(\omega) \leq x_i, & \forall i \in \mathcal{I}, \forall \omega \in \Omega \\
		& \quad x_i, y_j(\omega) \geq 0, & \forall i \in \mathcal{I}, j \in \mathcal{J}, \omega \in \Omega
	\end{aligned}
	\]
	
	with:
	
	\begin{itemize}
		\item $x_i$: decision variable representing the amount of component $i \in \mathcal{I}$ to produce (where $\mathcal{I}$ is the set of components);
		\item $y_j(\omega)$: amount of item $j \in \mathcal{J}$ assembled after demand realization (where $\mathcal{J}$ is the set of final items);
		\item $d_j(\omega)$: stochastic demand for item $j$ in scenario $\omega \in \Omega$.
		\item $C_i$: cost of component $i$;
		\item $P_j$: selling price of item $j$;
		\item $L_m$: availability of machine $m$;
		\item $T_{im}$: time required to produce component $i$ on machine $m$;
		\item $G_{ij}$: amount of component $i$ required to assemble item $j$ (Gozinto factor);
		\item $\mathcal{M}$: set of machines.
		
	\end{itemize}
	%\newpage
	In our specific implementation, we considered the example of the pizza maker, with eight hours of work available, two different types of pizzas to be able to produce and the following ingredients on hand: dough, tomato sauce, vegetables. The parameters in the optimization problem are set as follows:  
	\begin{itemize}
		\item $C = [3, 2, 2]$.
		\item $P = [7, 10]$.
		\item $T = [0.5, 0.25, 0.25]$.
		\item $L = 8.0$ hours.
		\item Gozinto matrix: $G = \begin{bmatrix} 1 & 1 \\ 1 & 1 \\ 0 & 1 \end{bmatrix}$.
	\end{itemize}
	
	\noindent These parameters and constraints form the basis of our computational experiments; they were chosen so as to produce results that could be reasonable and allow the problem to be solved even with a limited Gurobi license.
	\newpage
	\section{The class \texttt{ScenarioTree}}
	
	The scenario tree is a fundamental tool in stochastic optimization used to represent and manage uncertainty through a structured set of possible future scenarios. In this study, scenario trees are generated using a two-step process involving initial probabilistic models and a specialized Python class called \texttt{ScenarioTree}. This chapter contains the aspects of the above class most relevant to the assignment development.
	
	\subsection{Initial Probabilistic Model}
	
	Initially, scenarios are generated using a stochastic model defined by the class \texttt{EasyStochasticModel}. This class uses a multivariate normal distribution characterized by specified averages and variances; a fixed number of observations are sampled according to the probability distribution
	
	\[
	\text{Obs} \sim \mathcal{N}(\mu, \Sigma),
	\]
	
	whose parameters depend on the problem under consideration.	
	
	\subsection{Scenario Tree Generation}
	
	The \texttt{ScenarioTree} class is responsible for constructing and managing the tree structure. The tree is built iteratively, where each node generates child nodes according to the stochastic model described above. Each node has attributes:
	
	\begin{itemize}
		\item \textbf{obs}: observation values at the current node;
		\item \textbf{prob}: conditional probability of reaching the current node from its parent;
		\item \textbf{path\_prob}: cumulative probability from the root node to the current node.
	\end{itemize}
	
	Formally, for each node at stage $t$, child nodes are generated as follows:
	
	\[
	\text{obs}^{(t+1)}_j \sim \text{StochModel}\bigl(\text{obs}^{(t)}\bigr), \quad j=1, \dots, \text{branching\_factor}_t.
	\]
	
	The tree generation continues until the predefined depth (planning horizon) is reached, creating a comprehensive structure of all possible demand outcomes and associated probabilities. Two-stage problems were analyzed in this study, in which each of the generated scenario trees (of depth one) represents a set of possible realizations of the random variable demand. Finally, all the functions needed within the code to carry out the assignment were implemented within the \texttt{ScenarioTree} class.
	
	\
	
	\section{Results with all scenarios}
	
	In this section, are reported the results obtained by solving the stochastic optimization problems considering the full set of demand scenarios, without applying any reduction technique. This approach provides a benchmark solution, capturing the entire variability of the underlying random variables. The outcomes presented here will be useful as a reference for evaluating the accuracy and computational efficiency of the scenario reduction methods discussed in the following sections.
	
	
	\subsection{Newsvendor Problem}
	
	The following results summarize the optimization outcomes of the Newsvendor problem when considering all initially generated scenarios.\\
	
	\noindent To analyze the problem, 74 samples of not necessarily equal cardinality, but from the same distribution, were generated to simulate the behavior of the random variable demand. The fact that not all samples have the same cardinality highlights the different types of demand that can occur; nevertheless scenarios were generated with a maximum cardinality, related to the use of a limited Gurobi license. Next, 74 Newsvendor problems were solved, then as many samples of the expected value of profit were obtained. The 74 value was chosen so as to construct a $95\%$ confidence interval for the expected value of profit. \\
	
	\noindent The scenarios were generated as follows: 
	\begin{enumerate}
		\item Through the scenario tree class, from an initial fixed root, 74 nodes from the multivariate normal $\mathcal{N}_{40}(\mu, \Sigma)$ of size 40 were sampled, with $\mu = [25,\dots,25]$ and covariance matrix $\Sigma = 200 * I_{40}$. The choice of $\mu$ and $\Sigma$ parameters was dictated by the possibility of generating data that encapsulated some variability. 
		
		\item The scenario tree generated through this procedure returns demand values as floating-point numbers. However, since the Newsvendor problem inherently deals with discrete units (we are talking about newspapers), these continuous observations must be rounded to the nearest integer and constrained to be non-negative, ensuring practical and realistic demand scenarios. This rounding and aggregation procedure is implemented in the code using the provided \texttt{aggregate\_discrete\_demands} function, which combines scenarios with identical integer demands, summing their probabilities.
		
	\end{enumerate}
	
	
	\noindent In summary, the newsvendor problem is solved 74 times for each generated set, which as a result of aggregation will have cardinality less than or equal to 40. Table (\ref{tab:newsvendor-general}) shows an example of the output of a sample generated through the scenario tree, whose inputs were aggregated as necessary.\\
	
	\begin{table}[H]
		\centering
		\footnotesize % rende il testo più piccolo
		\label{tab:newsvendor-general}
		\renewcommand{\arraystretch}{1.1}
		\begin{tabular}{|@{\hskip 2pt}p{1.5cm}@{\hskip 2pt}|@{\hskip 2pt}p{2.0cm}@{\hskip 2pt}||@{\hskip 2pt}p{1.5cm}@{\hskip 2pt}|@{\hskip 2pt}p{2.0cm}@{\hskip 2pt}||@{\hskip 2pt}p{1.5cm}@{\hskip 2pt}|@{\hskip 2pt}p{2.0cm}@{\hskip 2pt}|}
			\hline
			\textbf{Demand (d)} & \textbf{Probability ($\pi$)} & \textbf{Demand (d)} & \textbf{Probability ($\pi$)} & \textbf{Demand (d)} & \textbf{Probability ($\pi$)} \\
			\hline
			0 & 0.06 & 19 & 0.08 & 39 & 0.06 \\
			1 & 0.02 & 21 & 0.02 & 42 & 0.02 \\
			4 & 0.02 & 23 & 0.04 & 43 & 0.02 \\
			5 & 0.02 & 24 & 0.02 & 46 & 0.04 \\
			7 & 0.02 & 25 & 0.04 & 48 & 0.02 \\
			10 & 0.02 & 29 & 0.04 & 56 & 0.04 \\
			12 & 0.02 & 30 & 0.04 & 57 & 0.02 \\
			13 & 0.06 & 31 & 0.02 & 59 & 0.02 \\
			14 & 0.02 & 33 & 0.02 & \textbf{Total} & \textbf{1.00} \\
			15 & 0.04 & 35 & 0.04 & & \\
			16 & 0.04 & 38 & 0.04 & & \\
			17 & 0.02 & & & & \\
			\hline
		\end{tabular}
		\caption{An example of discrete demand values and aggregated probabilities used in the Newsvendor model. First, an initial set with cardinality equal to 50 of equiprobabilistic scenarios was generated. Following the aggregation operation, the scenarios are reduced to 31, with probabilities no longer all equal.}
	\end{table}
	
	
	%\newpage
	\noindent The plot in Figure~(\ref{fig:scenariotree-plot}) illustrating an example of scenario tree  with the generated demand values, their probabilities, and the structure of uncertainty before aggregation.
	\begin{figure}[H]
		%\centering		
		\includegraphics[width=1\textwidth]{../immagini/scenariNV.png}
		\caption{Visualization of an example of scenario tree, used in the analysis of the Newsvendor problem. The tree is composed of 10 equiprobabilistic sets, each of dimension 5 sampled from a multivariate normal distribution of dimension 5 with parameters $\mu = 25, \sigma^{2} = 200$, before rounding and aggregation. The scenario tree clearly illustrating the branching structure, node values, and probabilities. }
		\label{fig:scenariotree-plot}
	\end{figure}
	
	
	\textcolor{red}{..Results..}
	\[
	\begin{array}{ccc}
		\hline
		\text{$\bar{p}_{N}$} & \text{$s_{N}^{2}$} & \text{$t_{S}$}\\
		\hline
		1 & 3 & 0.16 \\
		\hline
		%	\text{Total} & & 1.00 \\
		%	\hline
	\end{array}
	\]
	
	\label{tab:nv-results-stat}
	
	
	%	\newpage
	\subsection{ATO Problem}
	
	This subsection presents the results of the Assemble-to-Order (ATO) optimization model using the complete set of scenarios generated by our stochastic model.\\
	
	\noindent To analyze the problem, 85 samples of not necessarily equal cardinality, but from the same distribution, were generated to simulate the behavior of the random variable demand. The fact that not all samples have the same cardinality highlights the different types of demand that can occur; nevertheless scenarios were generated with a maximum cardinality, related to the use of a limited Gurobi license. Next, 85 Newsvendor problems were solved, then as many samples of the expected value of profit were obtained. The 85 value was chosen so as to construct a $95\%$ confidence interval for the expected value of profit. \\
	
	\noindent The scenarios were generated as follows: 
	\begin{enumerate}
		\item Through the scenario tree class, from an initial fixed root, 85 nodes of size 80 were sampled; each set consists of the succession of 40 pairs [Item 1,Item 2], where each pair represents the realization of a possible demand scenario. They are sampled from the multivariate normal $\mathcal{N}_{2}(\mu, \Sigma)$ with $\mu = [10,16]$ and covariance matrix $\Sigma =  \begin{pmatrix} 70 &0  \\ 0 &90  \end{pmatrix} $.The choice of $\mu$ and $\Sigma$ parameters was dictated by the possibility of generating data that encapsulated some variability, produce results that could be reasonable and allow the problem to be solved even with a limited Gurobi license. 
		
		\item The scenario tree generated returns demand values as floating-point numbers. However, since the Newsvendor problem inherently deals with discrete units (we are talking about newspapers), these continuous observations must be rounded to the nearest integer and constrained to be non-negative, ensuring practical and realistic demand scenarios. This rounding and aggregation procedure is implemented in the code using the provided \texttt{aggregate\_discrete\_demands} function, which combines scenarios with identical integer demands, summing their probabilities.
		
	\end{enumerate}
	
	\noindent In summary, also the ATO problem is solved 85 times for each generated set, which as a result of aggregation will be composed by a maximum number of realizations of demand scenarios equal to 40. Table (\ref{tab:ato-general-results}) shows an example of the output of a sample generated through the scenario tree, whose inputs were aggregated as necessary.
	%\newpage
	
	\noindent \textcolor{red}{Modificare tabella $\downarrow$}\\
	
	\begin{table}[H]
		\centering
		\footnotesize
		\label{tab:ato-general-results}
		\renewcommand{\arraystretch}{1.1}
		\begin{tabular}{|@{\hskip 2pt}p{3.2cm}@{\hskip 2pt}|@{\hskip 2pt}p{1cm}@{\hskip 2pt}||@{\hskip 2pt}p{3.2cm}@{\hskip 2pt}|@{\hskip 2pt}p{1cm}@{\hskip 2pt}||@{\hskip 2pt}p{3.2cm}@{\hskip 2pt}|@{\hskip 2pt}p{1cm}@{\hskip 2pt}|}
			\hline
			\textbf{Demand (Item 1, Item 2)} & \textbf{$\pi$} &
			\textbf{Demand (Item 1, Item 2)} & \textbf{$\pi$} &
			\textbf{Demand (Item 1, Item 2)} & \textbf{$\pi$} \\
			\hline
			\texttt{[60, 190]}  & 0.02 & \texttt{[80, 270]} & 0.02 & \texttt{[100, 100]} & 0.02 \\
			\texttt{[70, 160]}  & 0.04 & \texttt{[80, 280]} & 0.02 & \texttt{[100, 150]} & 0.02 \\
			\texttt{[80, 120]}  & 0.02 & \texttt{[90, 90]}   & 0.02 & \texttt{[100, 160]} & 0.04 \\
			\texttt{[80, 140]}  & 0.04 & \texttt{[90, 150]}  & 0.02 & \texttt{[100, 180]} & 0.04 \\
			\texttt{[80, 150]}  & 0.02 & \texttt{[90, 160]}  & 0.02 & \texttt{[100, 200]} & 0.04 \\
			\texttt{[80, 170]}  & 0.02 & \texttt{[90, 170]}  & 0.02 & \texttt{[100, 210]} & 0.02 \\
			\texttt{[80, 190]}  & 0.04 & \texttt{[90, 180]}  & 0.04 & \texttt{[100, 230]} & 0.02 \\
			\texttt{[80, 210]}  & 0.04 & \texttt{[90, 200]}  & 0.02 & \texttt{[100, 260]} & 0.02 \\
			\texttt{[80, 220]}  & 0.02 & \texttt{[90, 220]}  & 0.04 & \texttt{[100, 270]} & 0.04 \\
			\texttt{[80, 240]}  & 0.02 & \texttt{[90, 240]}  & 0.04 & \texttt{[110, 170]} & 0.02 \\
			\texttt{[80, 250]}  & 0.04 & \texttt{[90, 290]}  & 0.02 & \texttt{[110, 240]} & 0.04 \\
			\texttt{[80, 260]}  & 0.04 & \texttt{[90, 310]}  & 0.02 & \texttt{[110, 300]} & 0.02 \\
			\hline
			\multicolumn{5}{|r|}{\textbf{Total}} & \textbf{1.00} \\
			\hline
		\end{tabular}
		\caption{An example of discrete demand values and aggregated probabilities used in the ATO model. First, an initial set with cardinality equal to 40 of equiprobabilistic scenarios (each scenario is a two-dimensional vector) was generated. Following the aggregation operation, the scenarios are reduced to \textcolor{red}{$\rightarrow$ insert new number of scenarios} , with probabilities no longer all equal.}
	\end{table}
	
	
	%\newpage
	\noindent The plot in Figure~(\ref{fig:ato-scenariotree}) illustrating an example of scenario tree for an ATO problem  with the generated demand values, their probabilities, and the structure of uncertainty before aggregation.
	
	\begin{figure}[htbp]
		\centering
		\includegraphics[width=1\textwidth]{../immagini/scenariATO.png}
		\caption{Visualization of an example of scenario tree, used in the analysis of the ATO problem. The tree is composed of 10 equiprobabilistic sets each of dimension 4, composed of two pairs [Item 1,Item 2], each sampled from a multivariate normal distribution of dimension 2 with parameters $\mu = [10,16], \Sigma = \begin{pmatrix} 70 &0  \\ 0 &90  \end{pmatrix}$ before rounding and aggregation.}
		\label{fig:ato-scenariotree}
	\end{figure}
	
	\textcolor{red}{..Results..}
	\[
	\begin{array}{ccc}
		\hline
		\text{$\bar{p}_{N}$} & \text{$s_{N}^{2}$} & \text{$t_{S}$}\\
		\hline
		1 & 3 & 0.16 \\
		\hline
		%	\text{Total} & & 1.00 \\
		%	\hline
	\end{array}
	\]
	
	\label{tab:ATO-results-stat}
	
	\newpage
	\section{Results with K-means reduction}
	
	In this section, we present the results obtained by applying the K-means clustering method to reduce the number of scenarios and computational complexity generated in both the Newsvendor and Assemble-to-Order (ATO) problems. K-means is a clustering algorithm that in our implementation partitions the original set of scenarios into \(k\) clusters, by minimizing the weighted sum of squared distances between scenario values and their respective cluster centroids. Specifically, each original scenario is assigned to a cluster based on its proximity to the cluster's centroid, taking into account the scenario probabilities as weights. 
	%\newpage
	After the clustering process, the centroids become representative of the reduced scenarios, and the probabilities of these scenarios are recalculated by summing the probabilities of all original scenarios within each cluster.\\
	
	\noindent This procedure was implemented to reduce the number of original scenarios generated for both problems. The algorithm was applied to each of the N samples ($N = 74$ for NV, $N = 85$ for ATO), to reduce the numerosity of each sample to $k$, with $k \in [1,15]$. The upper bound of the interval was chosen such that the number of clusters obtained was significantly smaller than the original number of scenarios. Each resulting cluster, in fact, is represented by its centroid, which acts as a reduced scenario, with the new scenario probability obtained by summing the probabilities of the original scenarios assigned to that cluster. Furthermore, for each of the N samples, the SSE trend graph was plotted to identify the appropriate number of points (clusters) to represent each initial set, according to the algorithm. The SSE is the sum of squared Euclidean distances of each point to its closest centroid, so is a measure of error.\\
	
	\noindent Below, we present detailed analyses of the performance of this scenario reduction method in the two considered optimization problems.
	
	\subsection{Newsvendor Problem}
	
	In order to analyze the newsvendor problem with scenario reduction, $\forall k \in [1,15]$ and $N = 74$ the following were given in the table (\ref{tab:kmeans-nv-results}):
	\begin{itemize}
		\item $\bar{p}_{N}:$ estimation of the expected value of the profit with $95\%$ confidence interval, obtained by considering the sample mean computed using the $N$ values from solving the problem for each sample; 
		\item $s^{2}_{N}:$ sample variance of the expected value of the profit;
		\item $t_{R}:$ average computation time for reducing to $k$ scenarios;
		\item $t_{S}:$ average computation time for solving the Newsvendor problem with $k$ scenarios;
	\end{itemize}~
	
	\[
	\begin{array}{ccccc}
		\hline
		\text{\# Cluster $(k)$} & \text{$\bar{p}_{N}$} & \text{$s_{N}^{2}$} & \text{$t_{R}$} & \text{$t_{S}$}\\
		\hline
		1 & 3 & 0.16 & 0 & 0\\
		2 & 18 & 0.40 & 0 & 0\\
		3 & 34 & 0.26 & 0 & 0\\
		4 & 45 & 0.10 & 0 & 0\\
		5 & 57 & 0.08 & 0 & 0\\
		6 & 57 & 0.08 & 0 & 0\\
		7 & 57 & 0.08 & 0 & 0\\
		8 & 57 & 0.08 & 0 & 0\\
		9 & 57 & 0.08 & 0 & 0\\
		10 & 57 & 0.08 & 0 & 0\\
		11 & 57 & 0.08 & 0 & 0\\
		12 & 57& 0.08 & 0 & 0\\
		13 & 57 & 0.08 & 0 & 0\\
		14 & 57 & 0.08 & 0 & 0\\
		15 & 57 & 0.08 & 0 & 0\\
		\hline
		%	\text{Total} & & 1.00 \\
		%	\hline
	\end{array}
	\]
	\label{tab:kmeans-nv-results}
	
	\noindent \textcolor{red}{DA TENERE SE E' VERO: these results demonstrate how scenario reduction via K-means effectively simplifies the scenario representation while preserving the essential characteristics needed to achieve reliable decision-making outcomes in stochastic optimization contexts.}\\
	
	\noindent The following is a plot of the trend in SSE for each sample as the number of $k$ scenarios to which each set is reduced varies. The figure shows that, following the kmeans algorithm, the appropriate number of points to which to reduce the numerosity of each set is around 3 and 4, depending on the sample.
	
	%\newpage
	\begin{figure}[H]
		\centering
		\includegraphics[width=0.8\textwidth]{../immagini/sseNV.png}
		\caption{SSE trend graph for $k \in [1,15]$ in case of Newsvendor problem. The figure shows that the appropriate number of points to which to reduce the numerosity of each set is around 3 and 4.}
		\label{fig:sse-nv}
	\end{figure}
	
	
	%\newpage
	\subsection{ATO Problem}
	In this subsection, we present the results obtained for the Assemble-to-Order (ATO) problem after applying scenario reduction via K-means clustering, to identify representative scenarios in a two-dimensional space (corresponding to the two final products). In order to analyze the newsvendor problem with scenario reduction, $\forall k \in [1,15]$ and $N = 85$ the following were given in the table (\ref{tab:kmeans-ato-results}):
	\begin{itemize}
		\item $\bar{p}_{N}:$ estimation of the expected value of the profit with $95\%$ confidence interval, obtained by considering the sample mean computed using the $N$ values from solving the problem for each sample; 
		\item $s^{2}_{N}:$ sample variance of the expected values of the profit;
		\item $t_{R}:$ average computation time for reducing to $k$ scenarios;
		\item $t_{S}:$ average computation time for solving the Newsvendor problem with $k$ scenarios;
	\end{itemize}~
	
	\[
	\begin{array}{ccccc}
		\hline
		\text{\# Cluster $(k)$} & \text{$\bar{p}_{N}$} & \text{$s_{N}^{2}$} & \text{$t_{R}$} & \text{$t_{S}$}\\
		\hline
		1 & 3 & 0.16 & 0 & 0\\
		2 & 18 & 0.40 & 0 & 0\\
		3 & 34 & 0.26 & 0 & 0\\
		4 & 45 & 0.10 & 0 & 0\\
		5 & 57 & 0.08 & 0 & 0\\
		6 & 57 & 0.08 & 0 & 0\\
		7 & 57 & 0.08 & 0 & 0\\
		8 & 57 & 0.08 & 0 & 0\\
		9 & 57 & 0.08 & 0 & 0\\
		10 & 57 & 0.08 & 0 & 0\\
		11 & 57 & 0.08 & 0 & 0\\
		12 & 57& 0.08 & 0 & 0\\
		13 & 57 & 0.08 & 0 & 0\\
		14 & 57 & 0.08 & 0 & 0\\
		15 & 57 & 0.08 & 0 & 0\\
		\hline
		%	\text{Total} & & 1.00 \\
		%	\hline
	\end{array}
	\]
	\label{tab:kmeans-ato-results}
	
	
	
	\textcolor{red}{DA TENERE SE VERO: the scenario reduction via K-means preserved the essential characteristics of the demand distribution, allowing us to significantly reduce the problem size while maintaining optimality in the solution.}	\\
	
	\noindent The following is a plot of the trend in SSE for each sample as the number of $k$ scenarios to which each set is reduced varies. The figure shows that, following the kmeans algorithm, the appropriate number of points to which to reduce the numerosity of each set is around 3 and 4, depending on the sample.
	
	
	%\newpage
	\begin{figure}[H]
		\centering
		\includegraphics[width=0.8\textwidth]{../immagini/sseATO.png}
		\caption{SSE trend graph for $k \in [1,15]$ in case of ATO problem. The figure shows that the appropriate number of points to which to reduce the numerosity of each set is around 3 and 4.}
		\label{fig:sse-ato}
	\end{figure}
	
	\newpage
	\section{Results with Wasserstein distance--based reduction}
	
	In this section, we present the scenario reduction method based on the Wasserstein distance, as applied to both the Newsvendor and Assemble-to-Order (ATO) problems. The Wasserstein distance, also known as the Earth Mover's Distance, is a mathematical metric used to quantify the dissimilarity between two probability distributions on a given metric space. In the context of scenario reduction, it measures the minimum "cost" required to transform the original probability distribution of scenarios into a reduced one, where the "cost" is defined as the amount of probability mass to move times the distance it is moved.\\	
	
	\noindent Formally, let $\mu = (\mu_1, \ldots, \mu_m)$ and $\nu = (\nu_1, \ldots, \nu_n)$ be two discrete probability distributions on points ($x_1, \ldots, x_m$) and ($y_1, \ldots, y_n$). The $p$--Wasserstein distance is defined as:
	\[
	W_p(\mu, \nu) = \left( \min_{\gamma \in \Gamma(\mu, \nu)} \sum_{i=1}^m \sum_{j=1}^n \|x_i - y_j\|^p \gamma_{ij} \right)^{1/p}
	\]
	where $\gamma_{ij}$ is the transport plan representing the amount of mass moved from $x_i$ to $y_j$, and $\Gamma(\mu, \nu)$ is the set of admissible transport plans (satisfying mass conservation constraints). \\
	
	\noindent In our scenario reduction approach, we use an exact mixed-integer programming (MILP) formulation to select a subset of $k$ representative scenarios from the original $m$ scenarios, such that the Wasserstein distance between the original and reduced distributions is minimized. Below is reported the core optimization problem implemented in our code, in the unidimensional case:
	
	\[
	\begin{aligned}
		\min_{\gamma, z, \nu} \quad & \sum_{i=1}^m \sum_{j=1}^m c_{ij} \gamma_{ij} \\\\
		\text{s.t.} \quad
		& \sum_{j=1}^m \gamma_{ij} = \mu_i \qquad \forall i=1,\ldots,m \\
		& \sum_{i=1}^m \gamma_{ij} = \nu_j \qquad \forall j=1,\ldots,m \\
		& \nu_j \leq z_j \qquad \forall j=1,\ldots,m \\
		& \sum_{j=1}^m z_j = k \\
		& \sum_{j=1}^m \nu_j = 1 \\
		& z_j \in \{0,1\}, \quad \gamma_{ij}, \nu_j \geq 0
	\end{aligned}
	\]
	where:
	\begin{itemize}
		\item $c_{ij} = |x_i - x_j|^2$ is the cost of moving mass from scenario $i$ to scenario $j$ (Euclidean distance);
		\item $\gamma_{ij}$ is the amount of probability mass transported from $i$ to $j$;
		\item $z_j$ is a binary variable indicating whether scenario $j$ is selected in the reduced set;
		\item $\nu_j$ is the probability assigned to scenario $j$ in the reduced distribution.
	\end{itemize}
	\vspace{0.20cm}
	This model ensures that exactly $k$ scenarios are selected ($\sum_j z_j = k$), the reduced probabilities sum to 1, and the transportation of probability mass is minimized according to the cost matrix $C$. The same approach is extended to the multidimensional case (ATO problem), using the appropriate vector norms for the cost computation.
	The algorithm was applied to each of the 85 samples, to reduce the numerosity of each sample to $k$, with $k \in [1,15]$. Also for this problem, the upper bound of the interval was chosen such that the number of clusters obtained was significantly smaller than the original number of scenarios. \\
	
	\noindent Finally, it is important to point out that Wasserstein reduction technique, by construction, selects scenarios that preserve the probabilistic structure of the original distribution as faithfully as possible, yielding a reduced scenario set that guarantees a minimal loss of information with respect to the original distribution. In the following subsections, we detail the results obtained by applying this method to our stochastic optimization problems. 
	
	\subsection{Newsvendor Problem}
	
	For the Newsvendor problem, we applied the scenario reduction method based on the Wasserstein distance, which is designed to preserve the probabilistic structure of the original set of scenarios as closely as possible. The resulting reduced problems are summarized in Table~\ref{tab:wass-nv-results}, where $\forall k \in [1,15]$ and $N = 74$ the following were given:
	\begin{itemize}
		\item $\bar{p}_{N}:$ estimation of the expected value of the profit with $95\%$ confidence interval, obtained by considering the sample mean computed using the $N$ values from solving the problem for each sample; 
		\item $s^{2}_{N}:$ sample variance of the expected values of the profit;
		\item $t_{R}:$ average computation time for reducing to $k$ scenarios;
		\item $t_{S}:$ average computation time for solving the Newsvendor problem with $k$ scenarios;
	\end{itemize}~
	
	\[
	\begin{array}{ccccc}
		\hline
		\text{\# Cluster $(k)$} & \text{$\bar{p}_{N}$} & \text{$s_{N}^{2}$} & \text{$t_{R}$} & \text{$t_{S}$}\\
		\hline
		1 & 3 & 0.16 & 0 & 0\\
		2 & 18 & 0.40 & 0 & 0\\
		3 & 34 & 0.26 & 0 & 0\\
		4 & 45 & 0.10 & 0 & 0\\
		5 & 57 & 0.08 & 0 & 0\\
		6 & 57 & 0.08 & 0 & 0\\
		7 & 57 & 0.08 & 0 & 0\\
		8 & 57 & 0.08 & 0 & 0\\
		9 & 57 & 0.08 & 0 & 0\\
		10 & 57 & 0.08 & 0 & 0\\
		11 & 57 & 0.08 & 0 & 0\\
		12 & 57& 0.08 & 0 & 0\\
		13 & 57 & 0.08 & 0 & 0\\
		14 & 57 & 0.08 & 0 & 0\\
		15 & 57 & 0.08 & 0 & 0\\
		\hline
		%	\text{Total} & & 1.00 \\
		%	\hline
	\end{array}
	\]
	
	\label{tab:wass-nv-results}
	
	
	\textcolor{red}{DA TENERE SE E' VERO: These results demonstrate the effectiveness of Wasserstein reduction: although the number of scenarios is significantly decreased, the essential statistical features of the original demand distribution are maintained. The reduced scenario set still allows the optimization model to find a solution that is both robust and close to the original optimal profit.}	
	
	\subsection{ATO Problem}
	
	In this subsection, we report the results for the Assemble-to-Order (ATO) problem using scenario reduction via the Wasserstein distance. The original multidimensional demand scenarios were reduced to $k \in [1,15]$ representative scenarios by solving the exact MILP formulation that minimizes the Wasserstein distance, ensuring that the probabilistic and structural characteristics of the original distribution are preserved as faithfully as possible. The resulting reduced problems are summarized in Table\ref{tab:wass-ato-results}, where $\forall k \in [1,15]$  and $N = 85 $ the following were given:
	\begin{itemize}
		\item $\bar{p}_{N}:$ estimation of the expected value of the profit with $95\%$ confidence interval, obtained by considering the sample mean computed using the $N$ values from solving the problem for each sample;
		\item $s^{2}_{N}:$ sample variance of the expected values of the profit;
		\item $t_{R}:$ average computation time for reducing to $k$ scenarios;
		\item $t_{S}:$ average computation time for solving the Newsvendor problem with $k$ scenarios;
	\end{itemize}~
	
	\[
	\begin{array}{ccccc}
		\hline
		\text{\# Cluster $(k)$} & \text{$\bar{p}_{N}$} & \text{$s_{N}^{2}$} & \text{$t_{R}$} & \text{$t_{S}$}\\
		\hline
		1 & 3 & 0.16 & 0 & 0\\
		2 & 18 & 0.40 & 0 & 0\\
		3 & 34 & 0.26 & 0 & 0\\
		4 & 45 & 0.10 & 0 & 0\\
		5 & 57 & 0.08 & 0 & 0\\
		6 & 57 & 0.08 & 0 & 0\\
		7 & 57 & 0.08 & 0 & 0\\
		8 & 57 & 0.08 & 0 & 0\\
		9 & 57 & 0.08 & 0 & 0\\
		10 & 57 & 0.08 & 0 & 0\\
		11 & 57 & 0.08 & 0 & 0\\
		12 & 57& 0.08 & 0 & 0\\
		13 & 57 & 0.08 & 0 & 0\\
		14 & 57 & 0.08 & 0 & 0\\
		15 & 57 & 0.08 & 0 & 0\\
		\hline
		%	\text{Total} & & 1.00 \\
		%	\hline
	\end{array}
	\]
	
	\label{tab:wass-ato-results}
	
	
	\textcolor{red}{DA TENERE SE CORRETTO: This outcome highlights the robustness and accuracy of the Wasserstein-based scenario reduction, which manages to preserve the key features of the stochastic demand while significantly reducing computational complexity.
	}
	
	\newpage	
	\section{Efficiency}
	
	In this section, we compare the computational efficiency of the different scenario reduction methods applied to both the Newsvendor and ATO problems. We reported and discussed the execution times of each algorithm, summarizing the results through graphical visualizations. Times were calculated by averaging over the different values resulting for each sample.\\
	
	\noindent Figure~(\ref{fig:timing}) shows the total computational times required for the two problems, including both scenario reduction (with K-means and Wasserstein heuristic) and the subsequent optimization with the reduced scenarios. \textcolor{red}{$\downarrow$ insert new figure.}
	
	\begin{figure}[H]
		\centering
		\includegraphics[width=1\textwidth]{../immagini/tempiNV.png}
		\caption{Total execution times for Newsvendor problem and ATO problem, for each value of $k \in [1,15]$.}
		\label{fig:timing}
	\end{figure}
	
	
	\noindent The following table, on the other hand, shows the total run times both for Newsvendor problem and ATO problem, when all scenarios of each sample are considered, when the cardinality of each sample is reduced via Kmeans method, and finally, when the cardinality of each sample is reduced via Wasserstein  distance-based heuristic. \textcolor{red}{$\downarrow$ insert new table}.
	
	\begin{table}[htbp]
		\centering
		\begin{tabular}{|l|l|c|}
			\hline
			\textbf{Problem} & \textbf{Method} & \textbf{Total Time [s]} \\
			\hline
			Newsvendor & Full           & 0.0058 \\
			Newsvendor & K-means        & 0.1167 \\
			Newsvendor & Wasserstein    & 0.1324 \\
			\hline
			ATO        & Full           & 0.0059 \\
			ATO        & K-means        & 0.1207 \\
			ATO        & Wasserstein    & 0.1984 \\
			\hline
		\end{tabular}
		\label{tab:summary-times}
		\caption{Comparison between total execution times for Newsvendor problem and ATO problem with full scenarios, reduced scenarios via Kmeans and reduced scenarios via Wasserstein distance-based heuristic.}
		
	\end{table}
	
	\newpage
	
	\section{Discussion}
	
	\textcolor{red}{Modifica in base ai risultati.}\\
	
	\noindent From the chart, we observe that the time required to solve the full problem without reduction (\texttt{Full\_Solution}) is significantly lower than the time spent in the reduction phases (\texttt{KMeans\_Reduction} and \texttt{Wasserstein\_Reduction}), which dominate the total computational cost. Conversely, the actual optimization on the reduced scenario sets (\texttt{KMeans\_Solution} and 
	
	\texttt{Wasserstein\_Solution}) is almost negligible in terms of time. This result highlights that the bottleneck is the reduction procedure itself --especially when using methods such as K-means or Wasserstein MILP— while the optimization stage becomes extremely fast once the scenario set is reduced. Therefore, the choice of the reduction algorithm and its implementation has a direct impact on the overall computational efficiency of the workflow. . However, once the reduced scenario set is obtained, the final optimization becomes extremely efficient. This analysis confirms that, for moderate instance sizes, the major computational effort is concentrated in the scenario reduction phase—especially for methods based on mathematical programming—while the actual optimization benefits greatly from working on a smaller scenario set. Thus, efficiency considerations must take into account not only the quality of the reduced scenarios, but also the time required for the reduction procedure itself. The analyses presented in this report highlight important aspects regarding both the quality of solutions and the computational efficiency of different scenario reduction techniques when applied to the Newsvendor and Assemble-to-Order (ATO) problems.\\
	
	
	From a solution perspective, all scenario reduction methods—K-means and Wasserstein distance—were able to preserve the essential structure of the original problems. In both cases, the optimal decisions (order quantity for Newsvendor, component production plan for ATO) and the expected profit obtained with the reduced scenario sets were virtually identical to those derived from the full set of scenarios. This demonstrates the effectiveness of both K-means and Wasserstein-based reductions in maintaining solution quality while decreasing the scenario space.\\
	
	When considering computational efficiency, however, the results highlight important differences. The computational cost of the actual optimization, both for the full and reduced scenario sets, is negligible for both problems. The majority of the total computation time is instead spent on the scenarios reduction phase itself, with the Wasserstein MILP approach being the most time-consuming, especially for the multidimensional ATO case. K-means, while still significantly slower than the direct optimization, consistently requires less time than the Wasserstein-based reduction.\\
	
	\noindent Comparing the two problems, the trends remain similar: the scenario reduction phase dominates the total computational time--especially as the dimensionality of the problem increases--but both reduction methods succeed in dramatically simplifying the scenario set without any meaningful loss in solution quality. In summary, the choice of the reduction technique may be guided by the available computational resources and the problem's dimensionality: K-means offers faster execution and satisfactory accuracy for most practical purposes, while the Wasserstein approach is preferable when maximum fidelity to the original probability distribution is required, at the expense of increased computation time.
	
	%	\begin{table}[htbp]
		%		\centering
		%		\caption{Summary of optimal solutions and objective values for both Newsvendor and ATO problems.}
		%		\begin{tabular}{|l|l|c|c|c|}
			%			\hline
			%			\textbf{Problem} & \textbf{Method} & \textbf{\# Scenarios} & \textbf{Optimal Variable} & \textbf{Objective Value (€)} \\
			%			\hline
			%			Newsvendor & Full           & 31 & $x^* = 48$          & 202.00 \\
			%			Newsvendor & K-means        & 5  & $x^* = 45$          & 201.20 \\
			%			Newsvendor & Wasserstein    & 5  & $x^* = 42$          & 203.60 \\
			%			\hline
			%			ATO        & Full           & 36 & $x^* = [8,\,8,\,0]$ & 32.00  \\
			%			ATO        & K-means        & 5  & $x^* = [8,\,8,\,0]$ & 32.00  \\
			%			ATO        & Wasserstein    & 5  & $x^* = [8,\,8,\,0]$ & 32.00  \\
			%			\hline
			%		\end{tabular}
		%		\label{tab:summary-results}
		%	\end{table}
	
	
	
\end{document}